% === 切り替えコマンドの定義 ===
\newcommand{\switch}[2]{%switch{English}{Japanese} の形式で使う
  \ifjp#2\else#1\fi%
}



% === 言語データの定義 ===
\ifjp% 日本語用定義
  \newcommand{\OU}{大阪大学}
  \newcommand{\SANKEN}{産業科学研究所}
  \newcommand{\Affiliation}{\OU\SANKEN}
  \newcommand{\Address}{〒$565-0871$ 大阪府茨木市美穂ヶ丘 $8-1$}
  \newcommand{\Prof}[1]{#1 教授}
  \newcommand{\Ongoing}{ 現在}
  \newcommand{\Future}{将来}
  \newcommand{\AdvisorName}{指導教員}
  \DTMnewdatestyle{monthyear}{ % 日付フォーマット 引数は(年,月,日,曜日)
    \renewcommand*{\DTMdisplaydate}[4]{$##1$年$##2$月}
  }
  \newcommand{\PersonalWebsite}{kaki005.github.io/astro_academia/ja}
\else % 英語用定義
  \newcommand{\OU}{The University of Osaka}
  \newcommand{\SANKEN}{SANKEN}
  \newcommand{\Affiliation}{\SANKEN, \OU}
  \newcommand{\Address}{Mihogaoka $8-1$, Ibaraki, Osaka $567-0047$, Japan}
  \newcommand{\Prof}[1]{Prof.\! #1}
  \newcommand{\Ongoing}{ on}
  \newcommand{\Future}{future}
  \newcommand{\AdvisorName}{Supervisor}
  \DTMnewdatestyle{monthyear}{ % 日付フォーマット 引数は(年,月,日,曜日)
    \renewcommand*{\DTMdisplaydate}[4]{\DTMenglishmonthname{##2}, $##1$}
  }
  \newcommand{\PersonalWebsite}{kaki005.github.io/astro_academia/en}
\fi
\newcommand{\monthyear}[2]{%日付表示コマンド
  \DTMsetdatestyle{monthyear}%
  \DTMdisplaydate{#1}{#2}{-1}{-1}%
}

% ================================
% 言語共通の定義
% ================================
\newcommand{\Initials}{}
\newcommand{\Title}{Curriculum Vit\ae\ Summary}
\newcommand{\MyName}{
  \switch{Soshi Kakio}{垣尾 颯志}
}
\newcommand{\YSakurai}{
  \switch{Yasushi Sakurai}{櫻井 保志}
}
\newcommand{\YMatsubara}{
  \switch{Yasuko Matsubara}{松原 靖子}
}
\newcommand{\RFujiwara}{
  \switch{Ren Fujiwara}{藤原 廉}
}
\newcommand{\Me}{\underline{\MyName}}  % For citations
% URLs
\newcommand{\YSakuraiURL}{https://www.dm.sanken.osaka-u.ac.jp/~yasushi/index-j.html}
\newcommand{\YMatsubaraURL}{https://www.dm.sanken.osaka-u.ac.jp/~yasuko/}
\newcommand{\Email}{skakio88[at]sanken.osaka-u.ac.jp}

\newcommand{\LabWebsite}{www.dm.sanken.osaka-u.ac.jp}
\newcommand{\ORCID}{0009-0006-6200-3858}
\newcommand{\Linkedin}{soshi-kakio-659514297}
\newcommand{\GitHubProfile}{kaki005}
\newcommand{\Twitter}{kakikaki85}


% Macros to add links and mark publications
\newcommand{\DOI}[1]{doi:\href{https://doi.org/#1}{#1}}
\newcommand{\Website}[1]{\href{https://#1}{\url{#1}}}
\newcommand{\Preprint}[1]{Preprint: \href{https://doi.org/#1}{#1}}
\newcommand{\GitHub}[1]{\faGithub{} \href{https://github.com/#1}{#1}}
\newcommand{\Data}[1]{\faChartBar{} doi:\href{https://doi.org/#1}{#1}}


% Macros to set the year and duration on the left column
\newcommand{\Duration}[2]{\fontsize{10pt}{0}\selectfont {#1 -#2}}
\newcommand{\Year}[1]{\fontsize{10pt}{0}\selectfont {#1}}

% ================================
% コマンド
% ================================
% keyword
\definecolor{tagbg}{RGB}{225,236,244}
\definecolor{tagtxt}{RGB}{88,115,159}
\newcommand{\keytag}[1]{\tikz[baseline]{\node[anchor=base, rounded corners=0.5ex, text height=1.5ex, text depth=.25ex, fill=tagbg, draw=tagbg, text=tagtxt] {#1};}}

% Define a new environment to place all CV entries in a 2-column table.
% Left column are the dates, right column the entries.
\newcommand{\TablePad}{\vspace{-0.2cm}}
\NewEnviron{EntriesTableDuration}{
\TablePad
\begin{tabularx}{\textwidth}{@{}p{0.10\textwidth}@{\hspace{0.02\textwidth}}p{0.88\textwidth}@{}}
  \BODY
\end{tabularx}
\TablePad
}
\NewEnviron{EntriesTableYear}{
  \TablePad
  \begin{tabularx}{\textwidth}{@{}p{0.05\textwidth}@{\hspace{0.01\textwidth}}p{0.94\textwidth}@{}}
    \BODY
  \end{tabularx}
  \TablePad
}
\NewEnviron{EntriesTablePublications}{
\TablePad
  \begin{tabularx}{\textwidth}{@{}p{0.035\textwidth}@{\hspace{0.01\textwidth}}p{0.955\textwidth}@{}}
    \BODY
  \end{tabularx}
  \TablePad
}
\NewEnviron{EntriesTableRight}{
  \TablePad
  \begin{tabularx}{\textwidth}{@{}p{0.7\textwidth}@{\hspace{0.01\textwidth}}>{\raggedleft\arraybackslash}p{0.29\textwidth}@{}}
    \BODY
  \end{tabularx}
  \TablePad
}
